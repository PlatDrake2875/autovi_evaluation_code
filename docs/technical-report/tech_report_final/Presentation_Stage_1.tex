\documentclass{beamer}

\usetheme{Madrid}
\usecolortheme{default}

\usepackage[utf-8]{inputenc}
\usepackage[english]{babel}
\usepackage{amsmath}
\usepackage{amssymb}
\usepackage{graphicx}
\usepackage{booktabs}
\usepackage{tikz}
\usepackage{xcolor}

% Set colors for Madrid theme
\definecolor{politehnica_red}{RGB}{204, 0, 0}
\definecolor{politehnica_blue}{RGB}{0, 51, 102}
\setbeamercolor{palette primary}{bg=politehnica_blue, fg=white}
\setbeamercolor{palette secondary}{bg=politehnica_blue, fg=white}
\setbeamercolor{palette tertiary}{bg=politehnica_red, fg=white}

% Title page configuration
\title[Federated Anomaly Detection]{Federated Learning for Industrial Anomaly Detection}
\subtitle{Stage 1: Implementation and Results}

\author[Patrascu A. et al.]{
  Patrascu Adrian Octavian \and Cojocariu Raul \and Miriam Modiga
}

\institute[POLITEHNICA]{
  Faculty of Automatic Control and Computer Science \\
  National University of Science and Technology POLITEHNICA Bucharest
}

\date{November 26, 2025}

% Custom footer to prevent overflow
\setbeamertemplate{footline}
{
  \leavevmode%
  \hbox{%
    \begin{beamercolorbox}[wd=.333333\paperwidth,ht=2.25ex,dp=1ex,center]{author in head/foot}%
      \usebeamerfont{author in head/foot}Patrascu A., Cojocariu R., Modiga M.
    \end{beamercolorbox}%
    \begin{beamercolorbox}[wd=.333333\paperwidth,ht=2.25ex,dp=1ex,center]{title in head/foot}%
      \usebeamerfont{title in head/foot}POLITEHNICA Bucharest
    \end{beamercolorbox}%
    \begin{beamercolorbox}[wd=.333333\paperwidth,ht=2.25ex,dp=1ex,right]{date in head/foot}%
      \usebeamerfont{date in head/foot}\insertframenumber{}/\inserttotalframenumber\hspace*{2ex}
    \end{beamercolorbox}}%
  \vskip0pt%
}

\begin{document}

% ============================================================================
% SLIDE 1: Title Slide
% ============================================================================

\frame{\titlepage}

% ============================================================================
% SLIDE 2: Motivation & Key Challenges
% ============================================================================

\begin{frame}{Motivation \& Key Challenges}

\begin{block}{Why Federated Learning for Manufacturing?}
  \small
  \begin{itemize}
    \item Visual inspection critical for automotive quality control
    \item Traditional centralized ML faces significant deployment challenges
  \end{itemize}
\end{block}

\begin{block}{Three Key Challenges}
  \small
  \begin{enumerate}
    \item \textbf{Data Privacy}: Proprietary manufacturing process information
    \item \textbf{Data Silos}: Independent factories with isolated data repositories
    \item \textbf{Regulatory Constraints}: Quality control data transfer restrictions
  \end{enumerate}
\end{block}

\end{frame}

% ============================================================================
% SLIDE 3: Research Questions
% ============================================================================

\begin{frame}{Research Questions}

\begin{block}{Three Key Questions}
  \small
  \begin{enumerate}
    \item Can federated learning achieve competitive anomaly detection performance compared to centralized approaches?
    \item How does data heterogeneity affect federated model performance in industrial settings?
    \item What are the privacy-accuracy trade-offs in federated industrial inspection systems?
  \end{enumerate}
\end{block}

\vspace{0.5cm}

\begin{alertblock}{Stage 1 Focus}
  \small
  Stage 1 establishes baseline performance with independent client training (no aggregation yet)
\end{alertblock}

\end{frame}

% ============================================================================
% SLIDE 4: AutoVI Dataset
% ============================================================================

\begin{frame}{AutoVI Dataset}

\begin{columns}
\column{0.5\textwidth}
  \begin{block}{Dataset Overview}
    \begin{itemize}
      \item Real industrial dataset from Renault Group
      \item 6 automotive component categories
      \item 3,950 total images
      \item Pixel-level ground truth annotations
      \item Realistic production conditions
    \end{itemize}
  \end{block}

\column{0.5\textwidth}
  \begin{table}
    \centering
    \tiny
    \begin{tabular}{lcccc}
      \toprule
      \textbf{Category} & \textbf{Train} & \textbf{Test} & \textbf{Anom.} \\
      \midrule
      engine\_wiring & 285 & 607 & 322 \\
      pipe\_clip & 195 & 337 & 142 \\
      pipe\_staple & 191 & 305 & 117 \\
      tank\_screw & 318 & 413 & 95 \\
      underbody\_pipes & 161 & 345 & 184 \\
      underbody\_screw & 373 & 392 & 18 \\
      \midrule
      \textbf{Total} & \textbf{1,523} & \textbf{2,399} & \textbf{878} \\
      \bottomrule
    \end{tabular}
    \caption{\footnotesize AutoVI Statistics}
  \end{table}
\end{columns}

\end{frame}

% ============================================================================
% SLIDE 5: Federated Setup (6 Clients)
% ============================================================================

\begin{frame}{Federated Setup: Stage 1}

\begin{block}{Category-Based Client Partitioning}
  Each client represents a different production line specializing in one component type:
\end{block}

\begin{table}
  \centering
  \small
  \begin{tabular}{llcc}
    \toprule
    \textbf{Client} & \textbf{Category} & \textbf{Train} & \textbf{Test} \\
    \midrule
    1 & engine\_wiring & 285 & 607 \\
    2 & pipe\_clip & 195 & 337 \\
    3 & pipe\_staple & 191 & 305 \\
    4 & tank\_screw & 318 & 413 \\
    5 & underbody\_pipes & 161 & 345 \\
    6 & underbody\_screw & 373 & 392 \\
    \bottomrule
  \end{tabular}
  \caption{Stage 1: Independent Client Configuration}
\end{table}

\vspace{0.3cm}

\begin{block}{Stage 1 Approach}
  \begin{itemize}
    \item Each client trains \textbf{independently} on assigned category
    \item No aggregation in Stage 1 (establishes baseline)
    \item Total: 1,523 training images across 6 clients
  \end{itemize}
\end{block}

\end{frame}

% ============================================================================
% SLIDE 6: PatchCore Method - Architecture
% ============================================================================

\begin{frame}{PatchCore Method: Architecture}

\begin{columns}
\column{0.5\textwidth}
  \begin{block}{Feature Extraction}
    \begin{itemize}
      \item Pre-trained WideResNet-50-2 backbone
      \item Layer 2: [512, H/4, W/4] local features
      \item Layer 3: [1024, H/8, W/8] semantic features
      \item Concatenated: 1536-dimensional patches
    \end{itemize}
  \end{block}

\column{0.5\textwidth}
  \begin{block}{Memory Bank Construction}
    \begin{itemize}
      \item Greedy k-center coreset selection
      \item Keep 10\% of representative patches
      \item Ensures diverse feature space coverage
      \item Maintains computational efficiency
    \end{itemize}
  \end{block}
\end{columns}

\vspace{0.3cm}

\begin{block}{Anomaly Scoring}
  Distance to nearest neighbor in memory bank:
  \begin{equation*}
    s(x, p) = \min_{m \in M} \|f(x, p) - m\|_2
  \end{equation*}
  where $f(x,p)$ is the feature at patch position $p$ and $M$ is the memory bank.
\end{block}

\end{frame}

% ============================================================================
% SLIDE 7: PatchCore Method - Inference Pipeline
% ============================================================================

\begin{frame}{PatchCore Method: Inference Pipeline}

\begin{block}{From Image to Anomaly Map}
  \begin{enumerate}
    \item Extract multi-scale features (Layer 2 + Layer 3)
    \item Apply local neighborhood averaging (3×3 smoothing)
    \item Reshape to patch vectors
    \item Query FAISS index for nearest neighbor distances
    \item Upsample anomaly scores to original image resolution
    \item Compute image-level score (maximum of spatial map)
  \end{enumerate}
\end{block}

\begin{block}{Evaluation Metrics}
  \begin{itemize}
    \item \textbf{AUC-sPRO}: Pixel-level localization accuracy (at FPR: 0.01, 0.05, 0.1, 0.3, 1.0)
    \item \textbf{AUC-ROC}: Image-level classification accuracy
  \end{itemize}
\end{block}

\vspace{0.3cm}

\textit{\small Note: Feature extractor (WideResNet-50-2) is frozen; no training of network weights}

\end{frame}

% ============================================================================
% SLIDE 8: Implementation Status
% ============================================================================

\begin{frame}{Stage 1: Implementation Status}

\begin{alertblock}{Awaiting Results}
  Stage 1 experimental results will populate Tables 3 and 4 in technical report
\end{alertblock}

\vspace{0.2cm}

\begin{columns}
\column{0.48\textwidth}
  \begin{block}{\textcolor{green}{Completed} ✓}
    \small
    \begin{itemize}
      \item Data pipeline
      \item Feature extraction
      \item Memory bank
      \item FAISS indexing
      \item Evaluation metrics
    \end{itemize}
  \end{block}

\column{0.48\textwidth}
  \begin{block}{\textcolor{orange}{In Progress} ⏳}
    \small
    \begin{itemize}
      \item Baseline training
      \item Experimental results
      \item Performance analysis
      \item Visualization
    \end{itemize}
  \end{block}
\end{columns}

\end{frame}

% ============================================================================
% SLIDE 9: Expected Results Structure
% ============================================================================

\begin{frame}{Expected Results: Baseline Performance}

\begin{block}{Table 3: Centralized Baseline}
  \begin{table}
    \centering
    \tiny
    \begin{tabular}{lccccc}
      \toprule
      \textbf{Category} & \textbf{@FPR=0.01} & \textbf{@FPR=0.05} & \textbf{@FPR=0.1} & \textbf{@FPR=0.3} & \textbf{AUC-ROC} \\
      \midrule
      engine\_wiring & \textcolor{gray}{TODO} & \textcolor{gray}{TODO} & \textcolor{gray}{TODO} & \textcolor{gray}{TODO} & \textcolor{gray}{TODO} \\
      pipe\_clip & \textcolor{gray}{TODO} & \textcolor{gray}{TODO} & \textcolor{gray}{TODO} & \textcolor{gray}{TODO} & \textcolor{gray}{TODO} \\
      pipe\_staple & \textcolor{gray}{TODO} & \textcolor{gray}{TODO} & \textcolor{gray}{TODO} & \textcolor{gray}{TODO} & \textcolor{gray}{TODO} \\
      tank\_screw & \textcolor{gray}{TODO} & \textcolor{gray}{TODO} & \textcolor{gray}{TODO} & \textcolor{gray}{TODO} & \textcolor{gray}{TODO} \\
      underbody\_pipes & \textcolor{gray}{TODO} & \textcolor{gray}{TODO} & \textcolor{gray}{TODO} & \textcolor{gray}{TODO} & \textcolor{gray}{TODO} \\
      underbody\_screw & \textcolor{gray}{TODO} & \textcolor{gray}{TODO} & \textcolor{gray}{TODO} & \textcolor{gray}{TODO} & \textcolor{gray}{TODO} \\
      \bottomrule
    \end{tabular}
  \end{table}
\end{block}

\begin{block}{Expected Analysis}
  \begin{itemize}
    \item Categories with more training data → better performance
    \item Structural anomalies easier to detect than logical anomalies
    \item Performance variation across categories due to defect types
  \end{itemize}
\end{block}

\end{frame}

% ============================================================================
% SLIDE 10: Stage 2 Roadmap
% ============================================================================

\begin{frame}{Stage 2 Roadmap: Privacy \& Fairness}

\begin{columns}
\column{0.5\textwidth}
  \begin{block}{Privacy Enhancement}
    \begin{itemize}
      \item Differential Privacy (DP-SGD)
      \item Calibrated noise on local coresets
      \item Privacy budget tracking ($\epsilon$, $\delta$)
      \item Formal privacy guarantees
    \end{itemize}
  \end{block}

\column{0.5\textwidth}
  \begin{block}{Fairness: Cross-Category Equity}
    \begin{itemize}
      \item Fairness-aware aggregation weighting
      \item Per-client performance monitoring
      \item Address non-IID data heterogeneity
      \item Reduce performance variance
    \end{itemize}
  \end{block}
\end{columns}

\vspace{0.2cm}

\begin{block}{Combined Goal}
  \small
  Transform Stage 1 baseline into a \textbf{production-ready system} for real manufacturing environments where:
  \begin{itemize}
    \item Data privacy is paramount
    \item Regulatory compliance is required
    \item Cross-facility fairness is essential
  \end{itemize}
\end{block}

\end{frame}

% ============================================================================
% SLIDE 11: Conclusion
% ============================================================================

\begin{frame}{Conclusion}

\begin{block}{Stage 1 Achievements}
  \begin{enumerate}
    \item Complete data infrastructure for AutoVI dataset with federated partitioning
    \item PatchCore implementation (feature extraction, memory bank, FAISS inference)
    \item Federated system design with 6 independent clients
  \end{enumerate}
\end{block}

\begin{block}{Next Steps}
  \begin{itemize}
    \item Execute baseline training on all 6 clients
    \item Collect and analyze per-category performance metrics
    \item Generate anomaly map visualizations
  \end{itemize}
\end{block}

\begin{block}{Stage 2 Vision}
  Enhance Stage 1 with privacy-preserving mechanisms and fairness constraints to create a trustworthy federated learning system for industrial visual inspection.
\end{block}

\begin{center}
  \vspace{0.5cm}
  \Large \textbf{Questions?}
\end{center}

\end{frame}

% ============================================================================
% CLOSING SLIDE
% ============================================================================

\begin{frame}[plain]
  \begin{center}
    \vspace{1cm}
    \Huge Federated Learning for \\ Industrial Anomaly Detection

    \vspace{1cm}
    \Large Thank You!

    \vspace{1.5cm}
    \normalsize
    Patrascu Adrian Octavian \\ Cojocariu Raul \\ Miriam Modiga

    \vspace{0.5cm}
    \small
    POLITEHNICA Bucharest

    \vspace{1.5cm}
    \tiny
    \textit{For more information, see Technical Report: Stage 1}
  \end{center}
\end{frame}

\end{document}
