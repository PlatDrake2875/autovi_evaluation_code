\documentclass[11pt,a4paper]{article}

\usepackage[utf-8]{inputenc}
\usepackage[margin=1in]{geometry}
\usepackage{amsmath}
\usepackage{amssymb}
\usepackage{graphicx}
\usepackage{booktabs}
\usepackage{xcolor}
\usepackage{hyperref}
\usepackage{setspace}

\onehalfspacing

\title{Federated Learning for Industrial Anomaly Detection}

\author{
  Patrascu Adrian Octavian$^1$, Cojocariu Raul$^2$, Miriam Modiga$^3$ \\
  $^1$$^2$$^3$Faculty of Automatic Control and Computer Science \\
  National University of Science and Technology POLITEHNICA Bucharest
}

\date{\today}

\begin{document}

\maketitle

\begin{abstract}
This report presents Stage 1 of a federated learning system for visual anomaly detection in automotive manufacturing. We implement a comprehensive pipeline based on PatchCore, a state-of-the-art non-parametric anomaly detection method, on the AutoVI dataset containing 3,950 images across 6 industrial product categories. Our Stage 1 contributions include: (1) a complete data loading and preprocessing pipeline for the AutoVI dataset with category-based federated partitioning, (2) implementation of the PatchCore architecture including feature extraction via WideResNet-50-2, memory bank construction through greedy coreset selection, and FAISS-based efficient inference, and (3) federated system design with 6 independent clients (one per product category) establishing local baselines without aggregation. We present the implementation architecture, evaluation methodology using AUC-sPRO and AUC-ROC metrics, and detailed results structure for baseline performance across all categories. Stage 1 establishes the foundation for Stage 2 enhancements focusing on privacy preservation via differential privacy and fairness-aware aggregation mechanisms.

\textit{Keywords:} federated learning, anomaly detection, industrial inspection, PatchCore, privacy-preserving machine learning
\end{abstract}

\section{Introduction}

Industrial visual inspection is critical for ensuring product quality in manufacturing environments. Automotive production lines require robust defect detection systems capable of identifying anomalies that compromise vehicle safety and reliability. However, traditional centralized machine learning approaches face significant challenges in real-world industrial deployments.

Data privacy is a primary concern. Manufacturing data often contains proprietary information about production processes, quality metrics, and defect patterns that companies are reluctant to centralize. Additionally, different factories and production lines operate independently, creating isolated data repositories (data silos) that cannot be easily aggregated due to regulatory constraints. Industry regulations frequently restrict the transfer of quality control data across organizational boundaries.

\textit{Federated Learning} (FL) offers a compelling solution by enabling collaborative model training without centralizing sensitive data. In federated setups, multiple participants train local models on their private data and share only model updates with a central server for aggregation. This work investigates three key research questions: (1) Can federated learning achieve competitive anomaly detection performance compared to centralized approaches? (2) How does data heterogeneity affect federated model performance in industrial anomaly detection? (3) What are the trade-offs between privacy preservation and model accuracy in federated industrial inspection systems?

This Stage 1 report presents the foundational implementation with three main contributions: complete data infrastructure for the AutoVI dataset with category-based federated partitioning (1,523 training and 2,399 test images), PatchCore implementation following unsupervised anomaly detection principles, and federated architecture design with 6 independent clients establishing local baselines. Stage 1 establishes infrastructure for Stage 2 enhancements in privacy and fairness.

\section{Related Work}

Unsupervised anomaly detection in industrial quality control employs reconstruction-based methods (autoencoders, GANs), feature embedding approaches like PatchCore \cite{roth2022patchcore}, and knowledge distillation. PatchCore extracts features from pre-trained CNNs and stores representative normal patches in memory banks, detecting anomalies as patches distant from the memory bank. We adopt PatchCore for its strong performance, interpretability, and suitability for federated adaptation.

Federated Learning \cite{mcmahan2017communication} enables distributed training without raw data sharing. Key challenges include non-IID data distributions \cite{zhao2018federated}, communication costs, and privacy guarantees \cite{kairouz2019advances}. To our knowledge, no prior work applies federated learning to PatchCore-based industrial anomaly detection. Our memory bank aggregation approach presents a novel combination of communication efficiency and privacy preservation for manufacturing inspection systems.

\section{Dataset Description}

The \textit{Automotive Visual Inspection (AutoVI)} dataset \cite{autovidataset} is a genuine industrial dataset developed by Renault Group containing 6 object categories from automotive assembly with real production conditions. Image resolutions are category-specific: 400×400 pixels (small objects) and 1000×750 pixels (large objects).

\subsection{Dataset Statistics}

\begin{table}[h]
\centering
\small
\begin{tabular}{lcccccc}
\toprule
\textbf{Category} & \textbf{Train} & \textbf{Test} & \textbf{Good} & \textbf{Anom.} & \textbf{Types} \\
\midrule
engine\_wiring & 285 & 607 & 285 & 322 & 4 \\
pipe\_clip & 195 & 337 & 195 & 142 & 2 \\
pipe\_staple & 191 & 305 & 188 & 117 & 1 \\
tank\_screw & 318 & 413 & 318 & 95 & 1 \\
underbody\_pipes & 161 & 345 & 161 & 184 & 3 \\
underbody\_screw & 373 & 392 & 374 & 18 & 1 \\
\midrule
\textbf{Total} & \textbf{1,523} & \textbf{2,399} & \textbf{1,521} & \textbf{878} & \textbf{10} \\
\bottomrule
\end{tabular}
\caption{AutoVI Dataset Statistics. Training data contains only normal images; test data includes both normal and anomalous samples with pixel-level ground truth annotations.}
\end{table}

Defects are categorized as structural anomalies (physical damage, misalignment) and logical anomalies (missing/misplaced components) with pixel-level ground truth masks. Preprocessing includes resizing to category-specific dimensions, ImageNet normalization (mean=[0.485, 0.456, 0.406], std=[0.229, 0.224, 0.225]), and tensor conversion.

\subsection{Federated Data Partitioning}

For Stage 1, we implement category-based partitioning reflecting real industrial scenarios where different facilities handle different components:

\begin{table}[h]
\centering
\small
\begin{tabular}{lllcc}
\toprule
\textbf{Client} & \textbf{Category} & \textbf{Role} & \textbf{Train} & \textbf{Test} \\
\midrule
1 & engine\_wiring & Engine Assembly & 285 & 607 \\
2 & pipe\_clip & Clip Inspection & 195 & 337 \\
3 & pipe\_staple & Fastener Station & 191 & 305 \\
4 & tank\_screw & Fuel Tank & 318 & 413 \\
5 & underbody\_pipes & Underbody Line & 161 & 345 \\
6 & underbody\_screw & Underbody Fastening & 373 & 392 \\
\midrule
\textbf{Total} & \textbf{6 categories} & \textbf{-} & \textbf{1,523} & \textbf{2,399} \\
\bottomrule
\end{tabular}
\caption{Stage 1 Federated Partitioning. Each client operates independently on assigned category.}
\end{table}

\section{Methodology}

\subsection{PatchCore Architecture}

PatchCore is a non-parametric anomaly detection method that learns normality from defect-free training samples. The architecture comprises three components:

\textit{Feature Extraction.} A pre-trained WideResNet-50-2 backbone extracts multi-scale features from two layers: Layer 2 yields [512, H/4, W/4] local features; Layer 3 yields [1024, H/8, W/8] mid-level semantic features. Layer 2 is upsampled to match Layer 3 spatial dimensions, then concatenated, producing 1536-dimensional patch embeddings.

\textit{Memory Bank Construction.} The memory bank stores representative normal patch features using greedy coreset selection. The algorithm: (1) initializes with a random patch; (2) iteratively adds the patch maximizing minimum distance to the selected set; (3) continues until reaching target size (10\% of total patches). This ensures diverse coverage of the normal feature space while maintaining computational efficiency.

\textit{Anomaly Scoring.} During inference, anomaly scores are computed as the distance to nearest neighbor:

\begin{equation}
s(x, p) = \min_{m \in M} \|f(x, p) - m\|_2
\end{equation}

where $f(x, p)$ is the feature at patch position $p$ in image $x$, and $M$ is the memory bank. Scores are upsampled to pixel resolution using bilinear interpolation.

\subsection{Evaluation Metrics}

\textit{AUC-sPRO (Localization).} The saturated Per-Region Overlap (sPRO) metric measures pixel-level localization accuracy with saturation to prevent over-crediting large detections. We compute AUC-sPRO at multiple false positive rates: 0.01, 0.05, 0.1, 0.3, 1.0.

\textit{AUC-ROC (Classification).} Image-level anomaly classification uses the maximum anomaly score: $\text{image\_score}(x) = \max_{p} s(x, p)$. ROC curves are computed over normal vs. anomalous test images.

\section{Stage 1 Implementation and Results}

\subsection{Training Protocol and Expected Results}

Stage 1 establishes baseline performance with independent local training. Each of the 6 clients loads assigned category data, extracts features via WideResNet-50-2, builds local memory bank via greedy coreset selection (10\%), and evaluates on local test set. This approach provides category-specific baselines and establishes foundation for Stage 2 aggregation.

\textbf{TODO:} Centralized baseline results (AUC-sPRO at FPR thresholds and AUC-ROC for all categories):

\begin{table}[h]
\centering
\small
\begin{tabular}{lcccccc}
\toprule
\textbf{Category} & \textbf{@FPR=0.01} & \textbf{@FPR=0.05} & \textbf{@FPR=0.1} & \textbf{@FPR=0.3} & \textbf{AUC-ROC} \\
\midrule
engine\_wiring & TODO & TODO & TODO & TODO & TODO \\
pipe\_clip & TODO & TODO & TODO & TODO & TODO \\
pipe\_staple & TODO & TODO & TODO & TODO & TODO \\
tank\_screw & TODO & TODO & TODO & TODO & TODO \\
underbody\_pipes & TODO & TODO & TODO & TODO & TODO \\
underbody\_screw & TODO & TODO & TODO & TODO & TODO \\
\midrule
\textbf{Mean} & TODO & TODO & TODO & TODO & TODO \\
\bottomrule
\end{tabular}
\caption{Centralized Baseline Results. Awaiting PatchCore implementation completion.}
\end{table}

\textbf{TODO:} Per-client baseline performance (AUC-sPRO at select FPR thresholds and AUC-ROC for each client):

\begin{table}[h]
\centering
\small
\begin{tabular}{lllcccc}
\toprule
\textbf{Client} & \textbf{Category} & \textbf{Train} & \textbf{@FPR=0.01} & \textbf{@FPR=0.05} & \textbf{@FPR=0.1} & \textbf{AUC-ROC} \\
\midrule
1 & engine\_wiring & 285 & TODO & TODO & TODO & TODO \\
2 & pipe\_clip & 195 & TODO & TODO & TODO & TODO \\
3 & pipe\_staple & 191 & TODO & TODO & TODO & TODO \\
4 & tank\_screw & 318 & TODO & TODO & TODO & TODO \\
5 & underbody\_pipes & 161 & TODO & TODO & TODO & TODO \\
6 & underbody\_screw & 373 & TODO & TODO & TODO & TODO \\
\midrule
\textbf{Mean} & \textbf{-} & \textbf{1,523} & TODO & TODO & TODO & TODO \\
\bottomrule
\end{tabular}
\caption{Per-Client Independent Baseline Performance. Awaiting Stage 1 experimental completion.}
\end{table}

\section{Discussion}

\textbf{Stage 1 Status:} Data infrastructure is complete with all 6 categories loaded and partitioned. PatchCore implementation (feature extraction, memory bank, inference) has been developed. Federated client-server orchestration framework is ready. AUC-sPRO and AUC-ROC evaluation metrics are implemented. Awaiting experimental results from baseline training.

\textbf{Stage 2 Preview:} Future work will introduce privacy preservation via differential privacy (DP-SGD) with calibrated noise added to local coresets before server transmission. We will implement fairness-aware aggregation weighting to address non-IID data heterogeneity and per-category performance variance. This will transform Stage 1 into a production-ready system suitable for real manufacturing environments where data privacy, regulatory compliance, and cross-facility fairness are paramount.

\section{Conclusion}

This Stage 1 report establishes the foundation for a comprehensive federated anomaly detection system tailored to industrial visual inspection. By implementing PatchCore on the AutoVI dataset with category-based partitioning, we create a realistic framework for studying federated learning in manufacturing settings. Our architecture demonstrates that memory bank aggregation provides an efficient alternative to gradient-based federated learning, with potential communication and privacy benefits.

Stage 1 establishes baseline performance for independent local training. Stage 2 will introduce privacy and fairness enhancements, transforming this into a production-ready system suitable for real manufacturing environments where data privacy, regulatory compliance, and cross-facility fairness are paramount concerns.

\begin{thebibliography}{99}

\bibitem{roth2022patchcore}
Roth, K., Pemula, L., Zepf, J., Schölkopf, B., \& Brox, T. (2022).
Towards Total Recall in Industrial Anomaly Detection.
In \textit{IEEE/CVF Conference on Computer Vision and Pattern Recognition (CVPR)}.

\bibitem{autovidataset}
Carvalho, P., Pimentel, D., Bonfiglioli, R., \& Roth, K. (2023).
The Automotive Visual Inspection Dataset: AutoVI.
\textit{Zenodo}. https://doi.org/10.5281/zenodo.10459003

\bibitem{mcmahan2017communication}
McMahan, B., Moore, E., Ramage, D., Hampson, S., \& Arcas, B. A. (2017).
Communication-Efficient Learning of Deep Networks from Decentralized Data.
In \textit{Proceedings of the 20th International Conference on Artificial Intelligence and Statistics (AISTATS)}, pp. 1273--1282.

\bibitem{zhao2018federated}
Zhao, Y., Li, M., Lai, L., Suda, N., Civin, D., \& Chandra, V. (2018).
Federated Learning with Non-IID Data.
\textit{arXiv preprint arXiv:1806.00582}.

\bibitem{kairouz2019advances}
Kairouz, P., McMahan, H. B., Avent, B., Bellet, A., Bennis, M., Bhagoji, A. N., ... \& Zhao, S. (2019).
Advances and Open Problems in Federated Learning.
\textit{arXiv preprint arXiv:1912.04977}.

\bibitem{abadi2016deep}
Abadi, M., Chu, A., Goodfellow, I., McMahan, H. B., Mironov, I., Talwar, K., \& Zhang, L. (2016).
Deep Learning with Differential Privacy.
In \textit{2016 IEEE Symposium on Security and Privacy (SP)}. IEEE.

\end{thebibliography}

\end{document}
